\documentclass[conference]{IEEEtran}
% \IEEEoverridecommandlockouts
% The preceding line is only needed to identify funding in the first footnote. If that is unneeded, please comment it out.
\usepackage{cite}
\usepackage{amsmath,amssymb,amsfonts}
\usepackage{algorithmic}
\usepackage{graphicx}
\usepackage{textcomp}
\usepackage{xcolor}
%% \usepackage{natbib}

\def\BibTeX{{\rm B\kern-.05em{\sc i\kern-.025em b}\kern-.08em
    T\kern-.1667em\lower.7ex\hbox{E}\kern-.125emX}}
\begin{document}

\title{Privacy Preserving Selective Disclosure and Issuer Revocation of Verifiable Credentials*\\
%{\footnotesize \textsuperscript{*}Note: Sub-titles are not captured in Xplore and should not be used}
\thanks{Dyne.org Foundation}
}

\author{\IEEEauthorblockN{1\textsuperscript{st} Denis Roio}
\IEEEauthorblockA{
% \textit{dept. name of organization (of Aff.)} \\
\textit{Dyne.org Foundation}\\
Amsterdam, The Netherlands \\
jaromil@dyne.org}
\and
\IEEEauthorblockN{2\textsuperscript{nd} Andrea D'Intino}
\IEEEauthorblockA{
% \textit{dept. name of organization (of Aff.)} \\
\textit{The Forkbomb Company}\\
Copenhagen, Denmark \\
andrea@forkbomb.eu}
}

\maketitle

\begin{abstract}
It is of critical importance to design digital identity systems that ensure the privacy of citizens as well protect them from state corruption as the identity issuer. Unfortunately what is currently developed in Europe and USA is does not offer such basic protections. As a solution we introduce a method for unlinkable selective disclosure and privacy preserving revocation of digital credentials, utilizing the unique homomorphic characteristics 2nd order Elliptic Curves and Boneh-Lynn-Shacham (BLS) signatures operated on them. Our approach ensures that users can selectively reveal only the necessary attributes, while protecting their privacy across multiple presentations and against colluding verifiers. Since we also want to protect users from issuer corruption, we apply a threshold for credential issuance and revocation to mandate a collective agreement among multiple issuers. At last our method of revocation does not give out any information on the identity of holders of revoked credentials.
\end{abstract}

\begin{IEEEkeywords}
Privacy, Selective disclosure, BLS signatures, Digital credentials
\end{IEEEkeywords}

\section{Introduction}
Digital identity systems implement credential issuance and presentation mechanisms so that a person can voluntarily disclose his or her own acquired skills, professed attributes, or completed accomplishments. Credentials are signed by issuer authorities and encapsulated within various forms of digital proofs to be held in digital wallets, empowering individuals to reveal only chosen details to designated recipients, to limit data exposure and permit a user-controlled release of information.

Such systems are known as selective disclosure and this article aims at improving their cryptographic implementation to adhere to basic privacy-by-design standards.

\subsection{State of the art}

Selective disclosures are being used by nation states across the world in their next generation identity systems, for instance EIDAS2.0 in Europe where the European Digital Identity Wallet Architecture and Reference Framework mandates the use of the SD-JWT standard. Unfortunately the only SD-JWT implementation known and found in identity wallet implementations today adopts simple HMAC based cryptography to generate proofs.

In North America the situation seems to be different: the cryptography adopted is based on the BBS+ algorithm and applied to W3C Verifiable Credentials to obtain an higher degree of privacy.

The different choice of syntax in these two approaches is negligible, being Javascript Web Tokens of W3C Verifiable Credentials doesn't changes much from our point of view. But the cryptographic algorithms adopted have great importance for our purpose and they lack the features necessary to face three important threats which render them unsuitable to be used in real world situations.

\subsection{Threats considered}

\paragraph{Linkability}

Every presentation of an HMAC proof is identical and this makes it possible for colluding verifiers to trace an holder identity by following disclosures. In order to preserve the privacy of a credential holder the proofs disclosed by the wallet should not be traceable across different presentations. This threat appears to be well mitigated by BBS+ through its Zero Knowledge Proof implementation.

\paragraph{Lack of Revocation}

There is no revocation system designed, either for HMAC or BBS+. This may be mitigated by expiration dates, but in some cases they are not enough and interactive revocation is necessary. Since the choice is left open to developers, we will likely face huge privacy breaches with the adoption of revocation lists, as it has happened already during the COVID19 pandemic.

%% - https://github.com/ministero-salute/it-dgc-verificaC19-android/issues/103
%% - https://osf.io/preprints/lawarxiv/yc6xu


\paragraph{Issuer Corruption}

If the choice of interactive revocation is left to a single issuer, one may unilaterally choose to revoke credentials, without being subject to revision or having to seek consensus with a quorum of issuers. This situation leads to censorship and persecution of engaged individuals like journalists or activists living under dictatorial regimes that may arbitrarily revoke their credentials or even ID cards and passports.

\section{Overview}


\section{Implementation}

In this section we will provide a detailed description of the algorithm we propose for Selective Disclosure using BLS signatures (SD-BLS).

\subsection{Notations and assumptions}

We will adopt the following notations:
\begin{itemize}
    \item $\mathbb{F}_p$ is the prime finite field with $p$ elements (i.e. of prime order $p$); % In our case the prime is long 383 bits;
    \item $E$ denotes the (additive) group of points of the curve BLS-381-12 \cite{bls381-12} which can be described with the Weierstrass form  $y^2=x^3 + 16$;
    \item $E_T$ represents instead the group of points of the twisted curve of BLS-383, with embedding degree $k=12$. The order of this group is the same of that of $E$;
\end{itemize}
We also require defining the notion of a cryptographic pairing. Basically it is a function $e: \mathbb{G}_1\times\mathbb{G}_2\to \mathbb{G}_T$, where $\mathbb{G}_1,\mathbb{G}_2$ and $\mathbb{G}_T$ are all groups of same order $n$, such that satisfies the following properties:
\begin{itemize}
    \item [i.] \emph{Bilinearity}, i.e. given $P_1,Q_1\in\mathbb{G}_1$ and $P_2,Q_2\in\mathbb{G}_2$, we have
%    \item [i.] \emph{Bilinearity}, i.e. given $P,Q\in\mathbb{G}_1$ and $R,S\in\mathbb{G}_2$, we have
    \begin{align*}
        e(P_1+Q_1,P_2) = e(P_1,P_2)\cdot e(Q_1,P_2) \\
        e(P_1,P_2+Q_2) = e(P_1,P_2)\cdot e(P_1,Q_2)
    \end{align*}
    \item[ii.] \emph{Non-degeneracy}, meaning that for all $g_1\in\mathbb{G}_1, g_2\in\mathbb{G}_2$, $e(g_1,g_2)\ne 1_{\mathbb{G}_T}$, the identity element of the group $\mathbb{G}_T$;
    \item[iii.] \emph{ Efficiency}, so that the map $e$ is easy to compute;
    \item[iv. ] $\mathbb{G}_1\ne \mathbb{G}_2$, and moreover, that there exist no efficient homomorphism between $\mathbb{G}_1$ and $\mathbb{G}_2$.
\end{itemize}
For the purpose of our protocol we will have $\mathbb{G}_1 = E_T$ and $\mathbb{G}_2 = E$, and $\mathbb{G}_T\subset \mathbb{F}_{p^{12}}$ is the subgroup containing the $n$-th roots of unity, where $n$ is the order of the groups $E$ and $E_T$. Instead $e: E_T  \times E\to \mathbb{G}_T$ is the \emph{Miller pairing}, which in our work is encoded as the method \verb!miller(ECP2 P, ECP Q)!. \\

\bibliographystyle{IEEEtran}
\bibliography{references}

\end{document}
